% filepath: PID-in-Modern-Cherenkov/Sections/introduction.tex
% Content for the "Introduction" subsection

Modern high energy physics experiments rely on the performance of 
the detector systems to accurately study collision events. Cherenkov 
imaging detectors will play a central role in the upcoming generation of
experiments. Improving the capabilities of these detectors is the work
of thousands of physicists and engineers around the world.

This review aims to provide an introduction to the field of Cherenkov 
specific systems for particle identification (PID). We will cover the
basic principles of Cherenkov radiation, the design and operation of 
Cherenkov imaging detectors, and briefly over the latest 
advancements in technology. Cherenkov imaging detectors can be broadly 
divided into angle-imaging systems, such as RICH
and DIRC detectors used for PID at 
accelerators, and topology-imaging systems, such as large-volume water 
or ice Cherenkov detectors employed in neutrino and astroparticle 
physics.

We will begin with an overview of the theoretical background
in Sect \nolinebreak\ref{sec:theoretical-foundations}, and then introduce 
photomultipliers in Sect \nolinebreak\ref{sec:photomultipliers}. Following this, we will
delve into the main subclasses of Cherenkov imaging detectors in Sect \nolinebreak\ref{sec:cherenkov-imaging-detectors}, 
where over the chapters we will discuss their designs, principles of operation, 
and differences. Finally, we will explore the implementation of these detectors 
in upcoming experiments in Sect \nolinebreak\ref{sec:applications-future-facilities},
and conclude with a discussion of some open research in Sect \nolinebreak\ref{sec:current-challenges}.