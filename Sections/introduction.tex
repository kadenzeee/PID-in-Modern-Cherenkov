% filepath: PID-in-Modern-Cherenkov/Sections/introduction.tex
% Content for the "Introduction" subsection

\label{sec:intro}

Modern high energy physics experiments rely on the performance of 
the detector systems to accurately study collision events. Cherenkov 
based detectors will play a central role in the upcoming generation of
experiments. Improving the capabilities of these detectors is the work
of thousands of physicists and engineers around the world.

This review aims to provide an introduction to the field of Cherenkov 
specific systems for particle identification (PID). We will cover the
basic principles of Cherenkov radiation, the design and operation of 
the two main types of Cherenkov detectors, and briefly over the latest 
advancements in technology.

We will begin with an overview of the necessary theoretical background
in §\ref{sec:theoretical-foundations}, and then introduce photomultipliers, 
the ever-so important devices that detect photons in §\ref{sec:photomultipliers}. Following this, we will
delve into the two/\textcolor{red}{three} main types of Cherenkov detectors in §\ref{sec:rich} and
§\ref{sec:dirc} \textcolor{red}{and §\ref{sec:neutrino-identification}}, where over the two chapters we will discuss their designs,
principles of operation, and differences. Finally, we will explore the
implementation of these detectors in upcoming experiments in §\ref{sec:applications-future-facilities},
and conclude with a discussion of some open research in §\ref{sec:current-challenges}.