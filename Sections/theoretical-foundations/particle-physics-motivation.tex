% filepath: PID-in-Modern-Cherenkov/Sections/theoretical-foundations/particle-physics-motivation.tex
% Content for the "Particle Physics Motivation" subsection

\label{sec:particle-physics-motivation}

Detector assemblies are becoming ever more complex out of necessity, as the 
continual push for higher precision measurements dominates the investigation 
into quantum chromodynamics. Many types of particle physics experiments require particle identification (PID)
of hadrons such as pions, kaons and protons to observe rare processes or other heavy-ion
collisions. These PID requirements often span across extensive momentum and 
solid angle ranges; Cherenkov detectors are excellent at fulfilling these requirements.

For example, at the upcoming ePIC experiment at the Electron-Ion Collider (EIC), Cherenkov
methods will be used to provide hadron PID across essentially all high momentum ranges, as
shown in Fig \ref{fig:ePIC-pid-requirements}. These detector performance metrics will
be crucial to the experiment's physics goals towards investigating things such as nucleon 
spin, sea quarks, and high energy gluon behaviour.


\begin{figure}
\centering
\includegraphics[width=\linewidth]{Images/pid-requirements-ePIC.png}
\caption{\textbf{The ePIC PID requirements.} The pfRICH, hpDIRC and dRICH are all Cherenkov
based detectors that are expected to perform 3\(\sigma\) pion and kaon separation at momenta
between around 1 and 10 GeV\c. Here, \(\eta\) represents the pseudorapidity. (Photo: Greg Kalicy, ePIC Collaboration \cite{hpDIRC-Greg_2025})}\label{fig:ePIC-pid-requirements}
\end{figure}

To summarise, hadron detection is extremely important for flavour tagging and thus 
event reconstruction in these QCD processes. That is why Cherenkov detectors are 
imperative components of cutting edge detector assemblies.


