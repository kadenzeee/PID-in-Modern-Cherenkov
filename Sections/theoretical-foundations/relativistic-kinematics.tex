% filepath: PID-in-Modern-Cherenkov/Sections/theoretical-foundations/relativistic-kinematics.tex
% Content for the "Relativistic Kinematics" subsection

\label{sec:relativistic-kinematics}

Relativistic kinematics is the study of the motion of particles at near-light speeds. The purposes
of this review do not require an exhaustive treatment of the subject, but a few key concepts
pertaining to Cherenkov radiation and detector physics are summarised.

The famous energy-mass relation is written as:

\begin{equation}
E^2 = p^2c^2 + m_{0}^2c^4
\label{equ:energy-mass-relation}
\end{equation}

for a particle with energy \(E\), momentum \(p\) and rest mass \(m_{0}\). The Lorentz factor
arises from considering the four-velocity, and defines a mapping between the rest frame of the 
particle (the frame in which the particle's velocity is zero) and the observer's frame. It 
is defined as:
\begin{equation}
\gamma = \frac{1}{\sqrt{1 - \beta^2}}
\label{equ:lorentz-factor}
\end{equation}
where \(\beta = v/c\) is the velocity of the particle as a fraction of the speed of light. We 
can use the Lorentz factor to transform Newtonian quantities into their relativistic equivalents 
using the equations
\begin{align}
\label{equ:relativistic-momentum}
p &= \gamma m_{0} v      \\
\label{equ:relativistic-energy}
E &= \gamma m_{0} c^2   
\end{align} 

Using \eqref{equ:relativistic-momentum} and \eqref{equ:relativistic-energy} we can rewrite the Cherenkov angle condition from \eqref{equ:cherenkov-critical-angle} 
as
\begin{equation}
\cos{\theta} = \frac{c}{nv} = \frac{E}{npc} = \frac{\sqrt{p^2 + m^2c^2}}{np}
\label{equ:cherenkov-angle-relativistic}
\end{equation}
where one can clearer see how a combination of the Cherenkov angle and the momentum measurements can 
be used to infer the mass of the particle. This equation can be used to plot the theoretical curves
for the experimental data in Fig \ref{fig:cherenkov-angle-COMPASS}. 

Often in detector assemblies, the Cherenkov detectors will receive momentum measurements from forward detectors.
Then, the statistics from the Cherenkov detectors are so impressive, that the detectors are usually able to decisively 
improve on the forward detector's measurements. This is especially true for DIRC detectors, which have an
excellent angular resolution. The back-and-forth between detectors often creates friendly competition within
experimental collaborations.