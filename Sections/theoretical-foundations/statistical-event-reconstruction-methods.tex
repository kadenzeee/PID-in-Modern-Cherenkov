% filepath: PID-in-Modern-Cherenkov/Sections/theoretical-foundations/statistical-event-reconstruction-methods.tex
% Content for the "Statistical Event Reconstruction Methods" subsection

\label{sec:statistical-event-reconstruction-methods}

The problem of event reconstruction is to determine what happened inside the
detector based on the data obtained from the photo-sensors. The most common 
current way of doing this is as a reverse ray-tracing problem. An algorithm
iterates through the possible paths a photon can take through the detector,
and evaluates a Gaussian likelihood under the hypothesis of each particle type \cite{Santelj_2017}.

{\textcolor{red}{i want to put something about our time imaging algorithm here... help me roman! :D}}

In general, likelihood methods serve many detector assemblies well. However, they are in essence
a brute-force method, and as such plagued by the usual computation limitations.
One potential new-age solution is the use of machine learning. This aims to skip 
all of the setup and calculations surrounding the likelihood methods, and rather 
let a neural network learn how the detector reacts to certain particles. This has 
already seen encouraging results on Geant4 simulated data for the Hyper-Kaminokande
experiment \cite{Prouse_2023}.
