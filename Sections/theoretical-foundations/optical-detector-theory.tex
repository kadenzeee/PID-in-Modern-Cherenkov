% filepath: PID-in-Modern-Cherenkov/Sections/theoretical-foundations/optical-detector-theory.tex
% Content for the "Optical Detector Theory" subsection

Cherenkov detectors rely heavily on the principles of optics to guide and 
detect emitted photons. Key concepts include Snell's law, total internal reflection,
and the behaviour of light in various media. Fig \nolinebreak\ref{fig:panda-bd-event-display} shows a 
simulated event display of PANDA's Barrel DIRC detector \nolinebreak\cite{Singh_2019}, illustrating how
photons are internally reflected within the radiator bars and subsequently focused going into
the trapezoidal expansion volume before being detected by the photodetector array.

\begin{figure}
\centering
\includegraphics[width=\linewidth]{Images/panda-barrel-dirc-event-display.png}
\caption{\textbf{Simulated event display of PANDA Barrel DIRC.} The blue track is a charged 
particle incident on the detector, and the yellow tracks are all Cherenkov photons. There is
a mirror at the far end of the radiator bar, causing all photons to eventually reach the
detector array.}\label{fig:panda-bd-event-display}
\end{figure}

One can imagine how important the optical properties and the geometry of the detector are
to optimise things such as photon yield and angular resolution. All things told, the PANDA 
Barrel DIRC is actually a relatively rudimentary design compared to more recent simulations
for EIC DIRC detectors, which employ more complex focusing optics to improve performance. This
will be discussed further in Sect \textcolor{red}{sec:photon-optics} and Sect \textcolor{red}{sec:applications-at-future-facilities}. 
