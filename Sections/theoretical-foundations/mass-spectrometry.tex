% filepath: PID-in-Modern-Cherenkov/Sections/theoretical-foundations/mass-spectrometry.tex
% Content for the "Mass Spectrometry" subsection

Mass spectrometry refers to the mass measurement of a particle. There are many different methods,
such as time-of-flight (TOF), gas chromatography, quadrupole ion traps, etc. Mass spectrometry in
Cherenkov imaging detectors principally requires the measurement of the Cherenkov angle of a 
particle at a known momentum. Using the relativistic transformations for momentum and energy
\begin{align}
\label{equ:relativistic-momentum}
p &= \gamma m_{0} v      \\
\label{equ:relativistic-energy}
E &= \gamma m_{0} c^2   
\end{align} 
we can rewrite the Cherenkov angle from Equ \nolinebreak\ref{equ:cherenkov-angle} as
\begin{equation}
\cos{\theta_c} = \frac{\sqrt{p^2 + m^2c^2}}{np}\; .
\label{equ:cherenkov-angle-relativistic}
\end{equation}
The Cherenkov angle and the momentum can be used to infer the mass of the particle in a known medium. 
This equation can be used to plot the analytical curves for the experimental data in Fig \nolinebreak\ref{fig:cherenkov-angle-COMPASS}.

\begin{figure}
\centering
\includegraphics[width=\linewidth]{Images/cherenkov-angle.png}
\caption{\textbf{Cherenkov spectrometry of the COMPASS RICH-1 detector at CERN \nolinebreak\cite{Tessarotto_2014}.} There are clear curves for different 
particle species. The analytical form of these curves can be determined using Equ \nolinebreak\ref{equ:cherenkov-angle-relativistic}. 
Measurements of points on this curve can provide unique PID.\label{fig:cherenkov-angle-COMPASS}}
\end{figure}

In modern detector arrays at collider setups, forward detectors provide momentum measurements using 
TOF systems. Then, Cherenkov angle-imaging detectors (such as RICHs in Sect \nolinebreak\ref{sec:rich})
provide measurement of the Cherenkov angle. Topology-imaging detectors (such as Super-K in Sect \noelinebreak\ref{sec:neutrino-identification})
map the entire Cherenkov cone on photons using sensors over a huge area. This is not generally as accurate as angle-imaging, 
but the experimental programmes only demand mass differentiation between muons and electrons, rather than the precise mass
measurements in collider setups.

