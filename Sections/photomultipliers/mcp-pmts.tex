% filepath: PID-in-Modern-Cherenkov/Sections/photomultipliers/mcp-pmts.tex
% Content for the "MCP-PMTs" subsection

\label{sec:mcp-pmts}

Multichannel plate photomultipliers (MCP-PMTs) are devices that convert and 
amplify incoming photons into a measurable electrical signal. Photons incident
on a photocathode emit electrons through the photoelectric effect. These electrons
are difficult to measure as an electrical signal, so the number of electrons
is amplified using a multichannel plate. These are thin wafers that contain
thousands of \(\mu\)m capillaries. Via secondary emission, more electrons than 
are incident are produced \cite{Bruining_1954}. This amplified electron signal is then incident on 
an anode plate, where the signal is read as an analogue voltage.

\begin{figure}[h]
\centering
\includegraphics[width=\linewidth]{Images/mcp-pmt.png}
\caption{\textbf{Cross section of a traditional MCP-PMT.} Due to the proximity of
the anode and the photocathode, we are able to determine where the incident photon 
hit on the photocathode. MCP-PMTs are used in an incredible range of fields, 
from blood tests, to spectroscopy, to radar jamming. (Photo: \cite{photonics12010046})}\label{fig:traditional-mcp-pmt}
\end{figure}

In detector physics we are interested in a few properties of MCP-PMTs. Firstly,
the quantum efficiency (QE) of the photocathode is the probability of an incident photon
producing a photoelectron. Around 20\% are typical for modern MCP-PMTs. The requirements
for the PANDA Barrel DIRC ask for a QE of 18\% between 300 - 400nm \cite{panda-mcps-Katje_2025}. 

Secondly, the consistency of accurate hits is of great interest. We want uniform gain
across the entire MCP-PMT, and we do not want secondary electron emissions that cause faulty
signals. The former is referred to as the gain behaviour, or gain uniformity. The latter, where
we have erroneous secondary electron signals are called dark counts or afterpulsing depending 
on the cause. These are signals that are counted as a hit, but do not actually represent an
incident Cherenkov photon.

Timing characteristics are also very important. The timing resolution of the MCP-PMT refers to the 
uncertainty when the detector reports a single photon hit. Modern MCP-PMTs
are pushing 15ps time resolution \cite{Lyashenko_2026}. A lower timing resolution allows us to be more precise with
event reconstruction. The timing resolution is also somewhat inversely correlated to the rate 
capability of the detector; this is a measure of how many incident photons the MCP-PMT can handle
before becoming too saturated and therefore not being able to distinguish clear hits. The PANDA Barrel 
asks for a photoelectron rate capability of 500 \(\text{kHz/cm}^2\) \cite{panda-mcps-Katje_2025}.

Lastly, the lifetime of the MCP-PMT is of interest. If we are building detectors with price 
tags north of €100 million EUR, we want to make sure they last a good while. The main factors affecting 
MCP-PMT lifetime include overall exposure to photons, radiation damage and heat degradation/weathering. 
Minimising dark count rates and afterpulses is also crucial for MCP-PMT lifetime, as these add exposure 
and send electrons to places where they can cause chemical degradation.

MCP-PMT development is easily an entire thesis topic, so we will refrain from 
indulging more. However, it should be emphasised that this is a topic that is 
central to particle identification and detector physics; without MCP-PMTs, 
Cherenkov detectors are impossible.

