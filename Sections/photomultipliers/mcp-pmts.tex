% filepath: PID-in-Modern-Cherenkov/Sections/photomultipliers/mcp-pmts.tex
% Content for the "MCP-PMTs" subsection

Micro channel plate photomultipliers (MCP-PMTs) are devices that convert and 
amplify incoming photons into a measurable electrical signal. Photons incident
on a photocathode emit electrons through the photoelectric effect. These single electrons
are difficult to measure as an electrical signal, so the number of electrons
is amplified using a MCP. These are thin wafers that contain
thousands of \(\mu\)m capillaries. Via secondary emission, more electrons than 
are incident are produced \nolinebreak\cite{Bruining_1954}. This amplified electron signal is then incident on 
an anode plate, where the signal is read as an analogue voltage.

\begin{figure}[h]
\centering
\includegraphics[width=\linewidth]{Images/mcp-pmt.png}
\caption{\textbf{Cross section of a traditional MCP-PMT.} Due to the proximity of
the anode and the photocathode, we are able to determine where the incident photon 
hit. MCP-PMTs are used in an incredible range of fields, 
from blood tests, to spectroscopy, to radar jamming \nolinebreak\cite{photonics12010046}.}\label{fig:traditional-mcp-pmt}
\end{figure}

In detector physics we are interested in a few properties of MCP-PMTs. Firstly,
the quantum efficiency (QE) of the photocathode is the probability of an incident photon
producing a photoelectron; around 20\% is typical. 

Secondly, the consistency of accurate hits is of great interest. We want uniform gain
across the entire MCP-PMT, referred to as the gain behaviour. Higher gain means more distinct 
counts, and overall less uncertainty. We must also minimise faulty
signals from dark counts. Dark counts are spurious signals caused without
any incident photon, often due to thermionic emission or field emission. Balancing 
the gain behaviour, determined mostly by the anode bias, with minimising dark counts 
is a key challenge. Afterpulsing is where positive ions are created during the electron avalanche. These
positive ions accelerate backwards, towards the photocathode, causing potential damage and liberating 
secondary electrons. This creates a correlated, slightly delayed signal, appearing as a false photon hit.

Timing characteristics are also very important. The timing resolution of the MCP-PMT refers to the
uncertainty when the detector reports a single photon hit. Modern devices
are pushing 15ps time resolution \nolinebreak\cite{Lyashenko_2026}. A lower timing 
resolution allows us to be more precise with event reconstruction in detectors that rely on timing information. 
The timing resolution is also somewhat inversely correlated to the hit rate 
capability of the detector, a measure of how many incident photons the sensor can handle
before becoming too saturated.

Lastly, the lifetime of the MCP-PMT is of interest. The main factors affecting 
lifetime include overall exposure to photons, radiation damage and heat degradation. 
Minimising dark count rates and afterpulses is also crucial, as these send electrons 
to areas where they could cause chemical degradation.

MCP-PMT development is an extremely broad field spanning multiple disciplines; this overview 
just scratches the surface. The topic is central to PID and detector physics as a whole.

