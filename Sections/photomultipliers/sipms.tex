% filepath: PID-in-Modern-Cherenkov/Sections/photomultipliers/sipms.tex
% Content for the "SiPMs" subsection
% chktex-file 
\label{sec:sipms}

Silicon photomultipliers (SiPMs) have been gaining traction recently due to 
developments in materials, leading to a great ease of manufacturing. They are
tiny compared to an MCP-PMT, due to their use of 'microcells', each containing
an avalanche photodiode (APD) and a quenching resistor, to stop the avalanche.
The review \cite{ACERBI201916} provides an excellent in depth description of SiPMs.


\begin{figure}[h]
\centering
\includegraphics[width=\linewidth]{Images/sipm.png}
\caption{\textbf{Cross section of SiPM.} SiPMs are externally biased such that 
each avalanche photodiode (APD) is at breakdown voltage, allowing for electron
avalanches. The gain from an SiPM roughly matches that of a MCP-PMT at \(10^5\) - \(10^6\). (Photo: Hamamatsu \cite{hamamatsu_2016})}\label{fig:traditional-sipm}
\end{figure}

SiPMs are much cheaper than MCP-PMTs and have a higher quantum efficiency, however suffer from a high dark count rate 
(erroneous hits) and do not have the same rate capability as MCP-PMTs. This
has led to the explosion of SiPMs in fields like positron emission photography (PET)
where precise timings are not so necessary, and where ruggedness and sturdiness
is required. SiPMs also operate at a far lower voltage than MCP-PMTs and are not affected
by magnetic fields, adding to their dependability.
