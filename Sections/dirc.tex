% filepath: PID-in-Modern-Cherenkov/Sections/dirc.tex
% Content for the "Detection of Internally Reflected Cherenkov Light" subsection

Devices that detect internally reflected Cherenkov light are a type of Cherenkov detector, called DIRCs for short, circumvent the need 
for a separate radiator by using solid transparent bars, typically quartz, to both generate and guide 
Cherenkov photons to the photodetectors. Fig \nolinebreak\ref{fig:panda-bd-event-display} already
introduced in Section \nolinebreak\ref{sec:optical-detector-theory} showed a Geant4 simulation of the
PANDA Barrel DIRC detector \nolinebreak\cite{Singh_2019}, but Fig \nolinebreak\ref{fig:panda-dirc} shows a more
detailed rendering of the barrel detector itself. 

\begin{figure}[h]
\centering
\includegraphics[width=\linewidth]{Images/panda-barrel-dirc-cross-section.png}
\caption{\textbf{The PANDA Barrel DIRC along with an individual bar cross section.} 16 individual DIRC detectors are laid in a 
barrel-like arrangement, forming the very core of the detector array. Photons are internally reflected through the quartz bars.
}\label{fig:panda-dirc}
\end{figure}

The basic principle of DIRC operation is similar to that of other Cherenkov detectors; charged
particles incident on a medium with refractive index \(n>1\) emit Cherenkov photons in a cone,
and we try to measure the principle angle of this cone. However, in a DIRC detector the complexity
arises from the optics of trapping and guiding the Cherenkov photons to a small array of photodetectors.

\subsubsection{\label{sec:internal-reflection}Photon Transport via Internal Reflection}


Similarly to photon transport in optical fibres, the Cherenkov photons produced in the quartz bars of a DIRC 
are trapped via total internal reflection as shown in Fig \nolinebreak\ref{fig:panda-dirc}. The photons propagate to the end of the bar,
where they are focused into an expansion volume, and then onto the photodetector array. The DIRC relies on photons internally 
reflecting hundreds of times with minimal loss, all while preserving the Cherenkov angle precision. The fused silica bars provide 
extremely smooth surfaces to facilitate this, with surface roughness possible on the order of angstroms \nolinebreak\cite{fused-silica-smooth-TAN2023}.

The quartz bars are also UV transparent, allowing Cherenkov photons in the near-UV range to be detected, increasing the photon 
yield of the detector. Quartz is also known to be radiation hard, which will allow the detector to operate in the high radiation
environments of modern accelerators without significant degradation over time \nolinebreak\cite{radiation-hardness-silica-HOEK2008190}.
The fused silica bars also have a near optimal refractive index of around 1.47 in the visible spectrum, giving large Cherenkov angles.
Lower refractive indices would reduce the angle separation, harming PID performance, while higher refractive indices would 
increase chromatic dispersion, harming angular resolution. 


\subsubsection{\label{sec:timing-resolution}Chromatic Effects \& Timing Resolution}


Since Cherenkov photons are produced across a range of wavelengths, chromatic dispersion is unavoidable. Since the quartz bars 
are often meters long, blue photons often arrive 100s of picoseconds later than red photons. This is due to the wavelength dependence
of the refractive index, which is around 1.50 for blue light and 1.47 for red light in quartz. However, dispersion can be corrected for
if the photon arrival time is measured with sufficient precision, making it a crucial aspect of DIRC design. 

Modern MCP-PMTs are able to achieve single photon timing resolutions on the order of 15–20ps \nolinebreak\cite{Lyashenko_2026}. 
Improvements in readout electronics and signal processing are also pushing timing resolutions lower \nolinebreak\cite{electronics-readout-KEIZER2024169664}. 
These advancements allow photon timing to become a measurable observable, rather than just noise; detected photons provide \(x, y, t\) rather
than just \(x, y\). We can bake this information into our event reconstruction algorithms, simulating expected photon arrival times 
for bluer and redder photons separately, massively improving angular resolution. 

These sorts of chromatic corrections via timing information lead to far more sophisticated event reconstruction than RICH systems. As discussed in Section \nolinebreak\ref{sec:statistical-event-reconstruction-methods},
statistical methods such as maximum likelihood estimation are extremely powerful. DIRC systems offer a unique double-edged sword; complex photon paths, internal reflections,
and chromatic dispersion make analytical solutions near impossible, but the extremely detailed information available from precise timing measurements allow for robust
statistical models such as time-imaging \nolinebreak\cite{time-imaging-roman_2020} and machine learning to shine. 

\subsubsection{\label{sec:reconstruction-methods}Reconstruction Methods \& PID}

While DIRC systems provide excellent PID performance in compact geometries, their reliance on time-based global likelihood 
reconstruction from chromatic effects can complicate hit–track association in high-multiplicity collider environments compared to spatially imaging RICH 
detectors. The aforementioned time-imaging reconstruction method builds up a PDF from expected photon arrival times for all 
detector array channels and particle types. Then, events to be classified can be compared to these PDFs on a maximum likelihood basis. The algorithm 
can be optimised using a lookup table of possible paths a photon could take to a specific channel, eliminating a few unnecessary computations. 
This allows more flexibility in focusing optics within the detector geometry, as possible paths can be analytically determined and 
placed in the lookup table. This improves photon yield and therefore PID performance, up to 4.8 ± 0.1 \(\sigma\) \nolinebreak\cite{time-imaging-roman_2020}.
However, from a software standpoint, the algorithm is unfriendly to parallelise due to heavy use of conditional logic. 
Parallel hardware prefers to perform the same operation on many different data points. This algorithm demands different operations 
depending on detector channel, timing, and hypothesis. As such, computation time becomes a problem for large datasets. 

ML on the other hand is significantly easier to parallelise. Raw detector data is fed to the network on a per-event level and a sequence of 
fixed operations (mostly matrix multiplications) are performed. This inherently uses fewer CPU cycles per event when compared to time-imaging. The ML models are 
also easy to distribute onto modern hardware. Also since latency is predictable per event, as it is correlated to model size, data processing
is unlikely to `choke'; we are able to process batches of events at a time and can reliably expect all events to finish processing at 
around the same time. This is in contrast to variable latency such as in time-imaging reconstruction, where often batches are waiting on
one particularly difficult event to finish processing before starting the next batch. These advantages make ML scaleable in a way that 
maximum likelihood algorithms are not. 

Many different neural network architectures are being researched. This refers to the order and manner of the sequence of fixed operations 
that are done by the network. Dense networks consist of purely matrix multiplications, yet are seeing promising results \nolinebreak\cite{film-networks-markhoff_2026}.
More complicated networks such as normalising flows are closer to previous likelihood methods, but offer exact likelihood evaluation without
iterative maximisation, combining the known strong performance of the time-imaging reconstruction method with the compute advantages of 
ML methods \nolinebreak\cite{Fanelli_2025}. 
