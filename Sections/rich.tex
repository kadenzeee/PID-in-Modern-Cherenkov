% filepath: PID-in-Modern-Cherenkov/Sections/rich.tex
% Content for the "Ring Imaging Cherenkov Detectors (RICH)" subsection

\label{sec:rich}

Ring Imaging Cherenkov Detectors (RICH) are a class of Cherenkov detectors that use a radiator
at a known distance from a photodetector array to measure the Cherenkov angle of incident
particles. Shown in Fig \ref{fig:simple-rich} is a simplified Geant4 simulation of a RICH
detector. 

\begin{figure}[h]
\centering
\includegraphics[width=\linewidth]{Images/simple-rich-detector.png}
\caption{\textbf{Geant4 simulation of the simplified RICH detector.} The particle track (red) 
is incident on some material with a refractive index \(n\), called the radiator. In this case,
an aerogel radiator is simulated. The resulting Cherenkov photons (gray) emitted in a 
cone are incident on a square array of sensitive photodetectors. In practice, the detector 
array is often a ring, and there are optics surrounding the device to guide photons towards 
the detector.}\label{fig:simple-rich}
\end{figure}

The Cherenkov angles are emitted in a cone around the particle track, necessitating the ring
shape; having near light speed particles directly incident on photodetectors is likely to be
expensive. The outer radius of the ring is often encased in mirrors or lenses to trap Cherenkov
photons and guide them towards the photodetector array. These optical elements are carefully manufactured
to minimise chromatic dispersion and to maintain angular resolution.

\subsection{Radiators}

Different RICH experiments use different radiator materials. Common choices include aerogel
such as in the Belle II ARICH \cite{abe2010belleiitechnicaldesign}, gaseous radiators such as in
the old LHCb RICH \cite{LHCbRICH2000technicaldesign} and (although in mostly retired detectors) 
liquids such as in the barrel CRID at SLAC \cite{barrelCRID1999SLACtechnicalreport}.

Consider the Cherenkov angle condition from Eq \ref{equ:cherenkov-critical-angle}. The radiator
refractive index \(n\) is the key parameter that determines the choice of radiator, depending on 
expected beam momenta. Higher refractive indices are necessary for lower momentum particles, which
is why we see older accelerator systems using liquid radiators with inherently higher refractive indices.

The refractive index is also wavelength dependent, which is a source of chromatic dispersion. This is an
important consideration for radiator materials, as chromatic dispersion can degrade angular resolution. The
refractive index also affects the number of Cherenkov photons produced, as per the Frank-Tamm formula in 
Eq \ref{equ:frank-tamm-formula}. There are also tricks to play with the radiator design to improve photon
yield, such as layering aerogel with different refractive indices without performance cost \cite{IIJIMA2005383}. 
Naturally though complexity comes at a cost, especially in the case of finely engineered components such as
Cherenkov radiators.

\subsection{Optics}

Optical elements are often used in RICH detectors to increase photon yield. Lenses can be used to focus Cherenkov photons
onto smaller photodetector areas, reducing cost and size. Mirrors can be used to reflect photons that would otherwise
escape the detector. Other optical tricks such as the aforementioned layered radiator (also called proxy focusing) can be
used to tighten the spread of the photon cone, allowing lower momentum charged particles to be measured. 

As an applied example, the pfRICH at the forward section of the ePIC detector makes use of conical mirrors to increase 
the acceptance angle of particle tracks which can be seen in Fig \ref{fig:pfrich-conical-mirrors}. More photons means
better angular resolution, better event reconstruction and therefore more interesting physics. That's the idea anyway.

\begin{figure}[h]
\centering
\includegraphics[width=\linewidth]{Images/pfRICH-conical-mirrors.png}
\caption{\textbf{Schematic cross section of the pfRICH.} The conical mirrors (green) near the edge of the 
vessel mean that Cherenkov photons can be detected from steeper angles than would be traditionally possible. (Photo: Brian Page, eIPC Collaboration \cite{pfRICH-Brian_2025})}\label{fig:pfrich-conical-mirrors}
\end{figure}

\subsection{Mechanics \& Integration}

Modern detectors are large, complicated systems that enjoy premium real estate. Integrating a Cherenkov detector into such an 
array presents a unique set of mechanical and design challenges. Mechanical tolerances and alignment are crucial to maintaining
accuracy. Optical elements must be precisely aligned and consistently calibrated throughout the detector lifetime, requiring
robust mechanical support and space-hungry laser calibration systems. Careful thermal and radiation management
are also necessary to ensure the longevity of photosensitive components. Consideration of the surrounding magnetic field from
the solenoid used by the TOF detectors is also necessary, especially for PMTs which are magneto-sensitive. 

Ensuring that the detector can be serviced and maintained
is also a key consideration, especially for large scale experiments that may run for decades. Some experiments are built on
rails to allow the inner detectors (RICH, TOF, trackers etc.) the be slid out for maintenance.  
These mechanical and integration challenges are often underappreciated, but are imperative to the success of a detector.