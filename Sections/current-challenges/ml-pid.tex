% filepath: PID-in-Modern-Cherenkov/Sections/current-challenges/ml-pid.tex
% Content for the "ML PID" subsection

\label{sec:ml-pid}


Machine learning (ML) techniques are becoming increasingly prevalent in particle physics
as the need for fast and accurate data analysis grows. As detector arrays become larger,
so does the volume of data produced. Traditional PID methods, such as likelihood-based
event reconstruction as discussed in §\ref{sec:statistical-event-reconstruction-methods}
can become computationally expensive and time-consuming. ML algorithms when trained have
the possibility to skip all of the intermediate steps and directly classify particles 
from raw detector data. This is particularly useful in high-rate environments and is also
applicable to non-Cherenkov detectors.

The way that raw data is presented to the ML algorithm is crucial. Often we will see hit
patterns from photodetector arrays represented as 2D images, with pixel intensities represented
by colour scales. This makes convolutional neural networks (CNNs) a reasonable choice for ML PID,
as CNNs are well suited to image processing tasks. Other algorithms such as dense neural networks
(DNNs) and normalising flows \cite{Fanelli_2025} are also being explored.