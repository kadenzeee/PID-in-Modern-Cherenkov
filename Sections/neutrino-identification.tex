% filepath: PID-in-Modern-Cherenkov/Sections/neutrino-identification.tex
% Content for the "Neutrino Identification with Cherenkov Detectors" subsection



While Cherenkov detectors can not directly detect neutrinos as they are neutral particles,
neutrino events involving charged particles can be detected and reconstructed. However,
since these interactions are rare and large in scale, the detector and radiator material must span 
an enormous volume. The simple solution to this is to use easy or natural materials such as water or ice. This
presents a unique set of engineering challenges, such as waterproofing the detectors and synchronising timing
across large distances. It is also very difficult to separate the weakly interacting neutrino events from other
background. Different collaborations have taken unique approaches to these problems depending on their 
physics programmes.

\subsection{Water Cherenkov Detectors (Super-K)}
Water Cherenkov Detectors (WCDs) use large volumes of ultra-pure water as the radiator material. The Super-Kamiokande (Super-K)
experiment in Japan \nolinebreak\ref{fig:super-k} is a famous example of a WCD, using 50,000 tons of water surrounded by over 11,000 PMTs,
buried under a kilometre of mountain to shield from natural radiation. The detector is used to study neutrinos at MeV-GeV 
energies from various sources, including the sun, atmosphere, supernovae, allowing us to study neutrino-specific phenomena \nolinebreak\cite{super-k-neutrino-osc_1998},
which before Super-K were largely unexplored.

\begin{figure}[h]
\centering
\includegraphics[width=\linewidth]{Images/super-k.jpg}
\caption{\textbf{Inside the cylindrical water tank of Super-K.} Boats are used for the maintenance 
of water photodetectors. The detector is designed to identify neutrinos with MeV to GeV energies. 
(Photo: Kamioka Observatory, ICRR (Institute for Cosmic Ray Research), The University of Tokyo)}\label{fig:super-k}
\end{figure}

Super-K is squarely a precision detector for low energy neutrino physics. If we wish to study higher energy phenomena, such as 
cosmic neutrinos in the TeV-PeV range, we need to significantly increase the scale of the detector.

\subsection{Cosmogenic Neutrinos}

The IceCube Neutrino Observatory is a cubic kilometre scale Cherenkov detector buried deep in the Antarctic ice sheet. 
It uses over 5,000 photodetector modules arranged on 86 strings, deployed between 1.5 and 2.5 km deep. The detector is 
designed to identify high energy neutrinos from astrophysical sources. These neutrinos are filtered from low energy neutrinos
and other background by using the entire Earth as a shield; events that come from the core of the Earth are almost certainly high energy
neutrinos, as other particles would be absorbed. 

Fig \ref{fig:cosmic-neutrinos} depicts high energy cosmic neutrinos being formed through the Greisen–Zatsepin–Kuzmin (GZK)
process, where ultra high energy (>8 \textit{joules}) cosmic rays, accelerated by natural sources like black holes, interact with the cosmic 
microwave background radiation to produce pions, which then decay into neutrinos.

\begin{figure}[h]
\centering
\includegraphics[width=\linewidth]{Images/cosmic-neutrinos.png}
\caption{\textbf{Cosmic neutrinos formed through the GZK process.} Protons accelerated by black holes are able to 
interact with the cosmic background, decaying through intermediary steps into various high energy neutrinos. 
(Photo: Francis Halzen \cite{iceCUBE-Halzen_2025})}\label{fig:cosmic-neutrinos}
\end{figure}

Fig \ref{fig:ice-cube-event-display} shows such a cosmic event from IceCube, where a high energy neutrino has interacted with the ice
to produce muons and subsequent Cherenkov radiation. The figure is to give a sense of the scale of the detectors and the event topologies involved.
The thickness of the strings represents photon intensity, while the colours represent timing information, red being at the beginning of an event
and green/blue being later in time.

\begin{figure}[h]
\centering
\includegraphics[width=\linewidth]{Images/ice-cube-event-display.png}
\caption{\textbf{\textcolor{red}{figure for §\ref{sec:neutrino-identification}}} (Photo: Francis Halzen \cite{iceCUBE-Halzen_2025})}\label{fig:ice-cube-event-display}
\end{figure}

Cherenkov detectors have proven to be invaluable tools in the current study of physics, but they also have a promising future. We will now look 
at some case studies of upcoming detectors.