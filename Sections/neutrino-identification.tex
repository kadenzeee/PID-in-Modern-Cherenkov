% filepath: PID-in-Modern-Cherenkov/Sections/neutrino-identification.tex
% Content for the "Neutrino Identification with Cherenkov Detectors" subsection

\label{sec:neutrino-identification}

\subsection{Water Cherenkov Detectors}

\begin{figure}[h]
\centering
\includegraphics[width=\linewidth]{Images/super-k.jpg}
\caption{\textbf{\textcolor{red}{figure for §\ref{sec:neutrino-identification}}} (Photo: Kamioka Observatory, ICRR (Institute for Cosmic Ray Research), The University of Tokyo)}\label{fig:super-k}
\end{figure}

\subsection{Atmospheric Muons}

\begin{figure}[h]
\centering
\includegraphics[width=\linewidth]{Images/cosmic-rays.png}
\caption{\textbf{\textcolor{red}{figure for §\ref{sec:neutrino-identification}}} (Photo: Marzena Lapka, CERN)}\label{fig:cosmic-rays}
\end{figure}

\subsection{Cosmic Neutrinos}

IceCube Neutrino Observatory, use Earth as a filter to identify neutrinos from cosmic rays.

\begin{figure}[h]
\centering
\includegraphics[width=\linewidth]{Images/cosmic-neutrinos.png}
\caption{\textbf{\textcolor{red}{figure for §\ref{sec:neutrino-identification}}} (Photo: Francis Halzen \cite{iceCUBE-Halzen_2025})}\label{fig:cosmic-neutrinos}
\end{figure}

\begin{figure}[h]
\centering
\includegraphics[width=\linewidth]{Images/ice-cube-event-display.png}
\caption{\textbf{\textcolor{red}{figure for §\ref{sec:neutrino-identification}}} (Photo: Francis Halzen \cite{iceCUBE-Halzen_2025})}\label{fig:ice-cube-event-display}
\end{figure}